% !TeX root = ../install-latex-guide-zh-cn.tex

\chapter{在线 \LaTeX\ 平台}

在特定场合,
有些用户并不需要也没条件在本地安装发行版,
因此这里额外补充一些在线 \LaTeX\ 平台的相关内容,
如国外的 \href{https://www.overleaf.com/}{Overleaf},
和国内的 \href{https://www.texpage.com/}{TeXPage},
\href{https://www.loongtex.com/}{LoongTeX},
\href{https://www.slager.link/#/Home}{Slager}.
另外还有一些科研院所自己搭建的在线 \LaTeX\ 平台也可以在一定程度上满足用户的需要.

\section{Overleaf}

\subsection{注册 Overleaf}

Overleaf 是全球范围内首屈一指的在线 \LaTeX\ 编辑平台.
它为每位用户提供了 Ubuntu 系统下的 \TeX~Live.
它优秀的协作功能、丰富的模板仓库已吸引全球科研工作者成为它的用户.
Overleaf 提供了包括%
\href{https://cn.overleaf.com/}{中文}%
在内的多种语言供用户使用.
2024年6月27日,
\href{https://www.overleaf.com/blog/tex-live-2024-is-now-available}{它将后台的 \TeX~Live 升级为 2024 版本},
同时,
\href{https://www.overleaf.com/blog/new-feature-select-your-tex-live-compiler-version}{Overleaf 还允许用户自主选择项目中的 \TeX~Live 版本}.

目前 Overleaf 允许用户%
\href{https://www.overleaf.com/learn/latex/Chinese}{使用中文},
并为用户预先准备了一些%
\href{https://www.overleaf.com/learn/latex/Questions/Which_OTF_or_TTF_fonts_are_supported_via_fontspec%3F#Fonts_for_CJK}{中文字体}.
这些中文字体可通过%
\href{https://www.overleaf.com/latex/templates/using-the-ctex-package-on-overleaf-zai-overleafping-tai-shang-shi-yong-ctex/gndvpvsmjcqx}{C\TeX\ 宏集}%
调用.
特别推荐用户使用%
\href{https://www.overleaf.com/latex/examples/demonstration-of-noto-serif-cjk-and-noto-sans-cjk-fonts/sgrwgcddtqsq}{思源宋体和思源黑体}%
这两种开源中文字体.

遗憾的是,
国内网络环境会对 Overleaf 所使用的 reCaptcha 造成影响.
这也使得很多用户在直接注册 Overleaf 时就遇到了问题.

目前,
比较好的替代方案是借助 \href{https://orcid.org/}{ORCID} 来进行注册.
目前使用国内网络访问 ORCID 还比较流畅.
用户, 尤其是科研工作者, 可以先注册一个 ORCID 账号.
未来投稿时,
可将 ORCID 账号与自己的期刊网站账号进行绑定.
同时,
用户也可逐步将自己所发表论文列在 ORCID 网站以便管理.

\subsection{使用 Overleaf}

用户通过 ORCID 注册 Overleaf 后便可进入自己的项目列表页面进行使用.

新建项目是用户首先使用的功能.
Overleaf 提供了多种渠道为用户新建项目:
可以通过 Overleaf 中的模板,
也可以通过上传本地的 \texttt{zip} 压缩文件包.

新建项目后,
用户便可进入编辑界面.
在编辑界面用户需要先在左上角 \menu{Menu} 中选择合适的编译命令.
由于默认字体和文件编码等原因,
强烈建议用户在处理中文文档时使用 \hologo{XeLaTeX}.
编写文档后,
用户可通过鼠标点击按钮进行编译,
也可使用快捷键 \keys{ctrl + enter}.

另外,
Overleaf 在 2022 年 9 月 30 日更新了新的功能,
\href{https://www.overleaf.com/blog/new-feature-stop-on-first-error-compilation-mode}{Stop on first error},
这个功能一旦开启,
出现报错就立即停止编译.
它可以让用户高度关注报错信息并改正,
也可以帮助纠正一些会导致超时 (timeout) 的代码错误,
例如 Ti\textit kZ 里的 "\draw" 忘了最后的 ";".

用户编写的文件会保存在网站.
编写完成后,
用户只需点击 \menu{Menu} 旁的箭头回到项目列表.
这时可以看到新增项目右侧有四个图标,
它们分别是 \menu{Copy}、\menu{Download}、\menu{Archive} 和 \menu{Trash}.
用户可根据自己的需求点击合适的图标.

\subsection{选择不同发行版版本}

Overleaf 将后台 \TeX~Live 升级后,
用户新建项目默认使用 \TeX~Live 2024,
而老项目还是使用 \TeX~Live 的老版本.
如果用户不打算使用 \TeX~Live 2024,
只需在 \menu{Menu > Settings > TeX Live version} 选择版本即可.

\subsection{学习与帮助}

Overleaf 的%
\href{https://www.overleaf.com/latex/templates}{模板}%
和%
\href{https://www.overleaf.com/learn}{文档}%
对全网公开,
用户可以自行学习.
另外 Overleaf 有着专业的技术支援团队,
用户可发送邮件至
\href{mailto:support@overleaf.com}{\texttt{support@overleaf.com}}%
咨询使用过程中遇到的问题,
在邮件中请注意文明用语.
部分问题会超出免费服务的范畴,
用户需谨记这点.

\section{国内的在线 \LaTeX\ 平台}

相较于因网络原因而导致的难以注册和使用 Overleaf,
国内的在线 \LaTeX\ 平台则规避了网络问题,
因此本手册略微介绍一些,
方便用户选择.

\subsection{TeXPage}

用户只需访问 \href{https://www.texpage.com/}{TeXPage} 的主页,
使用邮箱注册即可,
具体过程不再赘述.

新用户注册 TeXPage 后会在页面看到一份使用教程.
这份教程简明扼要地概括了一般中文用户会遇到的常见问题,
然而从代码的角度而言,
我个人不欣赏在浮动体中使用 "H" 选项的做法.

在页面右上角的位置有许多菜单,
在\textsf{设置}菜单中,
用户可以选择默认的编译命令、发行版版本等等.
设置完毕,
再点击编译即可.

TeXPage 有自己的%
\href{https://www.texpage.com/docs}{文档中心},
文档数量不多,
但涵盖了相当一部分基础知识.
同时,
它也提供了联系方式 \href{mailto:support@texpage.com}%
{\texttt{support@texpage.com}},
用户如果遇到了一些使用上的问题也可以直接发邮件咨询.
当然用户在提问前也可以先尝试使用平台自带的 AI 调试和 AI 润色功能.

目前在 TeXPage 上有一些颇具特色的模板,
例如建模比赛的一些模板就已经被收录在 TeXPage 当中.
TeXPage 的%
\href{https://www.texpage.com/zh/pricing}{产品定价}%
涵盖了基础班、专业版、旗舰版和企业版,
用户可以按需购买.

\subsection{LoongTeX}

\href{https://www.loongtex.com/}{LoongTeX}
集成了白板、类 Notion 笔记管理、异步协作和 AI 辅助功能, 助力高效完成论文与文档.
新用户可通过微信、谷歌账号或邮箱快速%
\href{https://app.loongtex.com/user/login}{注册登录},
即刻进入专属工作台界面.

LoongTeX 工作台提供创建项目、新建笔记、发起白板三类核心创作入口,
创建项目支持三种高效启动方式: 上传 \texttt{zip} 压缩包、从%
\href{https://www.loongtex.com/templates/}{毕业论文库}%
生成、或绑定 Git 仓库实现云端同步协作.

新建项目后, 左侧导航栏集成文件树管理、
可视化 Git 版本历史及协作成员面板,
中央编辑区搭载智能 \LaTeX\ 编辑器与分屏 PDF 预览,
右侧功能面板提供 AI 助手和智能批注系统.

LoongTeX 官网内嵌%
\href{https://www.loongtex.com/docs/app/help/}{帮助中心},
提供分场景的 \LaTeX\ 写作指南、异步协作操作手册及AI助手使用技巧.
用户可发送邮件至
\href{mailto:loongtex@gmail.com}{\texttt{loongtex@gmail.com}}
咨询使用过程中遇到的问题.
\href{https://www.loongtex.com/pricing}{订阅方案}%
涵盖了专业版和团队版,
用户可以根据实际需要考虑购买.

\subsection{Slager}

\href{https://www.slager.link/#/Home}{Slager}%
支持邮箱注册,
目前看来,
Slager 的用法与其他在线平台并无太多不同,
\href{https://www.slager.link/#/HelpCenter}{帮助文档}%
也为新用户展示了初步的用法,
对于用户比较友好.

目前 Slager 的个人项目无数量限制,
并提供超大容量存储空间,
它内置的数学公式编辑器等等原本的会员权益,
如今通通免费使用,
并且已经取消了付费订阅模式.

\section{科研院所搭建平台}

大陆地区部分高校 (例如%
\href{https://overleaf.tsinghua.edu.cn/login}{清华大学}%
和%
\href{https://latex.ustc.edu.cn/login}{中国科学技术大学}) 搭建了供内部师生使用的平台.

中国科学院计算机网络信息中心科技云运行与技术发展部曾发邮件宣称开始提供论文协同编辑服务,
据我猜测,
它利用了早年 Sharelatex 的部分内容.
试用可点击%
\href{https://www.cstcloud.cn/resources/452}{这里}.
