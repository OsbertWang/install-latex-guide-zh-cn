% !TeX document-id = {9518a27b-c930-4bf6-a805-aa54f9dee147}
% !TeX TXS-program:compile=txs:///latexmk/{}[-pdfxe -synctex=1 -interaction=nonstopmode -file-line-error -silent %]

\documentclass[fontset=fandol]{ctexrep}
\usepackage{accsupp}
\usepackage[margin=2.4cm]{geometry}
\usepackage{listings}
\usepackage{xcolor}
\usepackage[pdfpagelayout=SinglePage]{hyperref}
\usepackage[os=win,hyperrefcolorlinks]{menukeys}

\ctexset{
  chapter ={
    format = \huge\bfseries\raggedright,
    number = \arabic{chapter},
    name   = {},
  },
  section/format = \Large\bfseries\raggedright
}

\lstset{
  backgroundcolor = \color{lightgray!30},
  keywordstyle    = \color{blue},
  stringstyle     = \color{brown!70},
  basicstyle      = {\small\ttfamily},
  breaklines      = true,
  tabsize         = 4,
  gobble          = 2,
  numbers         = left,
  numberstyle     = \tiny\emptyaccsupp,
  frame           = single,
  xleftmargin     = \ccwd,
  numbersep       = \ccwd,
  columns         = fullflexible,
%  emphstyle       = {\color{blue}\small\ttfamily},
%  emph            = {mkdir,rmdir,sudo,mount,umount,rm},
}

\newcommand\emptyaccsupp[1]{%
  \BeginAccSupp{ActualText={}}#1\EndAccSupp{}%
}

\renewmenumacro{\menu}[>]{angularmenus}
\renewmenumacro{\keys}[+]{shadowedroundedkeys}

\title{\bfseries 一份简短的安装 \LaTeX{} 的介绍\thanks{\url{https://github.com/OsbertWang/install_latex}}}
\author{啸行\thanks{\url{ranwang.osbert@outlook.com}}}
\date{\today}

\begin{document}
  
\maketitle

\begin{abstract}
在 QQ 群 91940767、478023327 和 640633524 时, 经常有群友咨询如何安装 \LaTeX.
本文将介绍在\textbf{未安装其他 \LaTeX{} 发行版的前提下} Windows 10、Ubuntu 和 macOS 系统中安装
\TeX{}~Live、升级宏包、编译简易文档、使用编辑器的相关操作, 并多以介绍命令行操作为主.
有关 Mik\TeX{} 的安装, 可以参考 \href{https://camuseblog.top/2019-03-02-/MiKTeX/}{MiK\TeX{} 的基本使用}.
建议用户阅读 \href{http://www.latexstudio.net/archives/11469.html}{\LaTeXe{} 安装 \& 新手指点 FAQ}
和 \href{http://mirrors.ctan.org/info/lshort/chinese/lshort-zh-cn.pdf}{lshort-zh-ch} 以更全面地了解基础内容.
本文还将简要介绍两款常见编辑器, 其他编辑器如 \href{https://github.com/EthanDeng/vscode-latex}{VS code}
和 \href{https://github.com/EthanDeng/sublime-text-latex}{Sublime Text}, 用户可自行了解它们的使用方法.
除在本地安装发行版之外, 本文还将额外补充使用 \href{http://www.overleaf.com}{Overleaf} 的相关内容.
本文所涉及到的代码需结合上下文说明, 不能简单地复制粘贴. 红色文字都是可点的超链接, 可直接跳转.
\menu{菜单} 表示软件菜单. \keys{k} 表示键盘按键.

本手册发布后,
Dongsheng Deng\footnote{\url{https://github.com/EthanDeng}},
muzimuzhi\footnote{\url{https://github.com/muzimuzhi}},
Xiangdong Zeng\footnote{\url{https://github.com/stone-zeng}}
对本手册提出了很好的建议, 并提供了帮助,
其中, 有关 macOS 的内容最初由 Xiangdong Zeng 草拟完成.
在此一并感谢.
\end{abstract}

\tableofcontents

% !TeX root = ../install-latex-guide-zh-cn.tex

\chapter{Windows 11 系统}

\section{安装 \TeX{} Live}\label{sec:windows:install}

在安装 \TeX{} Live 之前,
在文件夹菜单栏 \menu{查看 > 显示} 中选择 \menu{文件扩展名}.

用户可以从最近的 CTAN 源\footnote{这里所谓的最近, 其实是由系统判定的, 实际上系统可能会误判}下载 \TeX{} Live 的
\href{https://mirrors.ctan.org/systems/texlive/Images/texlive2023.iso}{iso 镜像文件},
也可以找大陆地区的源自行下载\footnote{新版本发布时, 各镜像网站同步的进度会有不同}.
通过大陆地区的源下载 iso 文件的方法请参考附录~\ref{chp:appendix:mirror}.

假设镜像文件被下载到本地后,
其存放路径为 \texttt{X:\textbackslash Y}%
\footnote{这里, \texttt{X} 表示盘符,
如 \texttt{C}、\texttt{D}、\texttt{E} 等;
\texttt{Y} 表示文件夹;
默认路径名称为不带空格的英文}.
下载完毕后, 用户打开 \textsf{cmd} 窗口%
\footnote{点击 \keys{\faWindows},
输入 \textsf{cmd},
点击 \keys{\enter};
或在任意文件夹中按 \keys{\ctrl + L},
输入 \textsf{cmd},
点击 \keys{\enter}},
执行
\begin{lstlisting}[language = bash]
  echo %path:;=&echo.%
\end{lstlisting}
查看环境变量. 
若 \texttt{C:\textbackslash Windows\textbackslash system32}
不在结果中\footnote{这里默认系统盘为\textsf{C}盘},
则关闭 \textsf{cmd} 窗口,
将 \texttt{C:\textbackslash Windows\textbackslash system32}
添加到环境变量中\footnote{在 \menu{桌面 > 此电脑 > 属性 > 高级系统设置 > 环境变量 > 系统变量 > Path} 中添加}.
\textbf{根据以往经验, 安装过 C\TeX{} 套装的用户往往会有} \texttt{system32} \textbf{丢失的问题}. 
另外, 如果环境变量中有 \texttt{mingw} 或 \texttt{jdk} 相关的内容,
也请暂时删除\footnote{具体原因参见
\href{https://tex.stackexchange.com/questions/445086/error-installing-latest-version-of-tex-live-on-windows-10}{stackexchange} 
的解释, 待安装结束后, 再将 \texttt{mingw} 或 \texttt{jdk} 环境变量添加至 \TeX{} Live 的环境变量之后}.
同时国内常见的 2345 好压也会对安装构成影响,
建议删除它并在安装 \TeX{} Live 后更换其他同类软件%
\footnote{推荐 \href{http://www.winrar.com.cn/}{WinRAR}
或 \href{https://www.7-zip.org/}{7-Zip}}.

下面我们要确保下载到的 iso 文件没有损坏, 一种常用做法是检查其 MD5 值.
当然, 如果确信自己下载到的文件没有问题, 也可以跳过这一步.
打开 \textsf{cmd} 窗口依次执行以下代码
\begin{lstlisting}[language = bash]
  cd /d X:\Y
  certutil -hashfile texlive2023.iso md5
\end{lstlisting}
若系统显示
\begin{lstlisting}
  MD5 的 texlive2023.iso 哈希:
  787f087e71695eebd1caafdb2b286060
  CertUtil: -hashfile 命令成功完成.
\end{lstlisting}
则表明下载的镜像文件正常.
除 MD5 值外,
还可检查 SHA512 值.
依然是在 \textsf{cmd} 执行
\begin{lstlisting}[language = bash]
  cd /d X:\Y
  certutil -hashfile texlive2023.iso sha512
\end{lstlisting}
若系统显示
\begin{lstlisting}[language=bash,literate={a}{a}{1} {b}{b}{1} {c}{c}{1} {d}{d}{1} {e}{e}{1} {f}{f}{1}]
  SHA512 的 texlive2023.iso 哈希:
  3a78cc59d562a9224543b90318ecb28d3ae7f975c2a031e15304a20b8fb0abac6e0ed63012c0bdf8f5edd39380caf122a17c948c05b28ea38fe90f2f0b19bdba
  CertUtil: -hashfile 命令成功完成.
\end{lstlisting}
即为验证成功.

接下来开始安装.
推荐用户先手动建立 \TeX{} Live 的安装路径 (文件夹) 再进行安装,
注意路径的名称是\textbf{不带空格的英文},
默认路径为 \texttt{C:\textbackslash texlive\textbackslash 2023}.
假如系统用了中文作为用户名,
那么用户需要先按照 \ref{sec:chinesename} 节的方法更改临时环境变量后再安装.
将镜像文件装载至虚拟光驱\footnote{Windows 11 默认双击镜像文件便可装载;
受某些压缩软件影响双击失效时,
可在文件夹菜单栏
\menu{装载};
第三方虚拟光驱软件可考虑 WinCDEmu 或 UltraISO;
不鼓励解压缩镜像文件}. 
假设加载后的路径为 \texttt{Z:\textbackslash}.
在卸载了国内第三方安全软件的前提下,
打开 \textsf{cmd} 窗口并执行
\begin{lstlisting}[language = bash]
  Z:\install-tl-windows.bat --no-gui
\end{lstlisting}
即可看到如下结果
\begin{lstlisting}
  =====================> TeX Live installation procedure <=====================
  
  ======>   Letters/digits in <angle brackets> indicate   <=======
  ======>   menu items for actions or customizations      <=======
  = help>   https://tug.org/texlive/doc/install-tl.html   <=======
  
  Detected platform: Windows (64-bit)
  
  <B> set binary platforms: 1 out of 6
  
  <S> set installation scheme: scheme-full
  
  <C> set installation collections:
      41 collections out of 41, disk space required: 7851 MB
  
  <D> set directories:
    TEXDIR (the main TeX directory):
      C:/texlive/2023
    TEXMFLOCAL (directory for site-wide local files):
      C:/texlive/texmf-local
    TEXMFSYSVAR (directory for variable and automatically generated data):
      C:/texlive/2023/texmf-var
    TEXMFSYSCONFIG (directory for local config):
      C:/texlive/2023/texmf-config
    TEXMFVAR (personal directory for variable and automatically generated data):
      ~/.texlive2023/texmf-var
    TEXMFCONFIG (personal directory for local config):
      ~/.texlive2023/texmf-config
    TEXMFHOME (directory for user-specific files):
      ~/texmf
  
  <O> options:
    [ ] use letter size instead of A4 by default
    [X] allow execution of restricted list of programs via \write18
    [X] create all format files
    [X] install macro/font doc tree
    [X] install macro/font source tree
    [X] adjust search path
    [1] add menu items, shortcuts, etc.
    [1] update file associations
    [X] install TeXworks front end
    [X] after install, set CTAN as source for package updates
  
  <V> set up for portable installation
  
  Actions:
  <I> start installation to hard disk
  <P> save installation profile to 'texlive.profile' and exit
  <Q> quit
  
  Enter command:
\end{lstlisting}
这时, 用户可可直接按键盘 \keys{I} 在默认路径中直接安装 \TeX{} Live,
也可按键盘 \keys{D} 来更改安装路径.
按 \keys{D} 后, 用户可看到
\begin{lstlisting}
  ==============================================================================
  Directories customization:
  
  <1> TEXDIR:       C:/texlive/2023
      main tree:    C:/texlive/2023/texmf-dist
  
  <2> TEXMFLOCAL:     C:/texlive/texmf-local
  <3> TEXMFSYSVAR:    C:/texlive/2023/texmf-var
  <4> TEXMFSYSCONFIG: C:/texlive/2023/texmf-config
  
  <5> TEXMFVAR:       ~/.texlive2023/texmf-var
  <6> TEXMFCONFIG:    ~/.texlive2023/texmf-config
  <7> TEXMFHOME:      ~/texmf
  
  Note: ~ will expand to %USERPROFILE%
  
  Actions:
  <R> return to main menu
  <Q> quit
  
  Enter command:
\end{lstlisting}
这时按键盘 \keys{1}, 将看到
\begin{lstlisting}
  New value for TEXDIR [C:/texlive/2023]:
\end{lstlisting}
这时可更改路径, 如 \texttt{D:/texlive/2023}.
接下来, 按键盘 \keys{R} 即可回到初始的安装界面,
再按键盘 \keys{I} 便可在刚才指定的路径安装.
用户可通过阅读英文执行其他操作, 这里不再赘述. 
至此, 用户只需耐心等待安装完成, 并且不要点击 \textsf{cmd} 窗口%
\footnote{Windows 11 默认采取快速编辑模式, 一旦点击 \textsf{cmd} 窗口,
将显示``选择命令提示符''而导致进程暂停}. 
安装完成后, 关闭再打开 \textsf{cmd} 窗口%
\footnote{正常安装 \TeX{} Live 后, 环境变量会变化,
关闭原 \textsf{cmd} 窗口再重新打开才能获取新的环境变量},
执行
\begin{lstlisting}[language = bash]
  tex -v
\end{lstlisting}
显示
\begin{lstlisting}
  TeX 3.141592653 (TeX Live 2023)
  kpathsea version 6.3.5
  Copyright 2023 D.E. Knuth.
  There is NO warranty.  Redistribution of this software is
  covered by the terms of both the TeX copyright and
  the Lesser GNU General Public License.
  For more information about these matters, see the file
  named COPYING and the TeX source.
  Primary author of TeX: D.E. Knuth.
\end{lstlisting}
即可认为安装顺利完成%
\footnote{若未能正确显示结果, 则可能是因为环境变量未改变,
用户可手动添加 \texttt{<TEXDIR>\textbackslash bin\textbackslash windows} 至环境变量,
\texttt{<TEXDIR>} 需与你方才的安装路径一致}.
弹出前面装载的光盘镜像,
安装到此结束.

\subsection{下载镜像一直不成功的处理办法}

目前发现有些用户使用某些浏览器时无法正常下载镜像文件.
这时,
用户可以选择使用 \href{https://eternallybored.org/misc/wget/}{\textsf{wget}} 对镜像进行下载.
具体用法是:
将 \textsf{wget} 所在目录添加到环境变量中.
打开 \textsf{cmd} 窗口执行以下代码
\begin{lstlisting}[language = bash]
  wget http://mirrors.ctan.org/systems/texlive/Images/texlive2023.iso
\end{lstlisting}
即可下载最近的镜像.
类似地,
也可以指定国内的源进行下载,
例如选择清华大学镜像
\begin{lstlisting}[language = bash]
  wget https://mirrors.tuna.tsinghua.edu.cn/CTAN/systems/texlive/Images/texlive2023.iso
\end{lstlisting}
其他国内镜像只需要将链接替换即可,
链接和方法请参考附录~\ref{chp:appendix:mirror}.

\subsection{需要管理员权限的处理办法}

有些用户反馈安装过程中需要管理员权限.
以管理员身份打开 \textsf{cmd} 窗口的方法是点击 \keys{\faWindows},
输入 \textsf{cmd},
按 \keys{\ctrl + \shift + \enter} 三个键.
一般而言,
在系统盘内直接安装 \TeX{} Live 会碰到此类问题,
因此鼓励用户在安装前先手动建立安装路径.
直接使用管理员身份打开 \textsf{cmd} 窗口进行安装可能会对以后的升级造成影响.

\subsection{系统用中文作为用户名}\label{sec:chinesename}

很多中文 Windows 用户习惯于用中文作为用户名,
但是这样会导致 \TeX{} Live 安装失败.
更改系统的 \texttt{TEMP} 和 \texttt{TMP} 这两个环境变量可以避免这样的问题.
在 \textsf{cmd} 中执行
\begin{lstlisting}
  mkdir C:\temp
  set TEMP=C:\temp
  set TMP=C:\temp
\end{lstlisting}
便可更改这两个环境变量为 \texttt{C:\textbackslash temp}.
更改后按照前述内容安装 \TeX{} Live 即可.
这样的更改是临时的,
待 \textsf{cmd} 窗口关闭,
环境变量将恢复为原始状态.

\section{卸载 \TeX{} Live}

在跨版本升级 \TeX{} Live 时, 通常需要卸载旧版 \TeX{} Live.
下面介绍几种卸载方法.

\subsection{使用批处理文件}

在命令行执行
\begin{lstlisting}[language=bash]
  kpsewhich -var-value TEXMFROOT
\end{lstlisting}
可以查看 \texttt{TEXMFROOT} 的值,
该值即为 \TeX{} Live 的安装路径%
\footnote{%
  除这里提供的 \texttt{TEXMFROOT}, 还有很多其他变量, 可在
  \href{https://www.tug.org/texlive/doc/texlive-zh-cn/texlive-zh-cn.pdf}{texlive-zh-cn}
  第 2.3 节处找到
}. 
接下来,
用户只需执行%
\footnote{双击或在 \texttt{cmd} 中运行}%
安装路径中的卸载批处理文件即可实现卸载, 如默认安装时执行
\begin{lstlisting}[language=bash]
  C:\texlive\2023\tlpkg\installer\uninst.bat
\end{lstlisting}
结束以上步骤后,
手动删除用户文件夹中的 \texttt{.texlive2023}.
如果用户安装 \TeX{} Live 时需要管理员权限,
那么在卸载时同样需要管理员权限.

卸载后,
用户需要清理注册表
\begin{lstlisting}
  HKEY_CURRENT_USER\Software\Classes\TL.bitmap2eps.convert.2023
  HKEY_CURRENT_USER\Software\Classes\TL.DVIOUT.view.2023
  HKEY_CURRENT_USER\Software\Classes\TL.TeXworks.edit.2023
  HKEY_CURRENT_USER\Software\Classes\TL.PSViewer.view.2023
  HKEY_CURRENT_USER\SOFTWARE\Microsoft\Windows\CurrentVersion\ApplicationAssociationToasts\TL.bitmap2eps.convert.2023_.jpg
  HKEY_CURRENT_USER\SOFTWARE\Microsoft\Windows\CurrentVersion\ApplicationAssociationToasts\TL.bitmap2eps.convert.2023_.png
  HKEY_CURRENT_USER\SOFTWARE\Microsoft\Windows\CurrentVersion\ApplicationAssociationToasts\TL.TeXworks.edit.2023_.sty
  HKEY_CURRENT_USER\SOFTWARE\Microsoft\Windows\CurrentVersion\ApplicationAssociationToasts\TL.TeXworks.edit.2023_.tex
\end{lstlisting}
但若在安装时使用管理员权限``为所有人安装'',
则注册表位置将变为
\begin{lstlisting}
  HKEY_LOCAL_MACHINE\SOFTWARE\Classes\TL.bitmap2eps.convert.2023
  HKEY_LOCAL_MACHINE\SOFTWARE\Classes\TL.DVIOUT.view.2023
  HKEY_LOCAL_MACHINE\SOFTWARE\Classes\TL.TeXworks.edit.2023
  HKEY_LOCAL_MACHINE\SOFTWARE\Classes\TL.PSViewer.view.2023
\end{lstlisting}

\subsection{手动卸载}

如果执行批处理文件 \texttt{uninst.bat} 出错,
用户也可手动删除安装文件夹,
之后再清理 \TeX{} Live 的环境变量,
并且清理注册表.

\section{跨版本升级 \TeX{} Live}

目前在 \href{https://www.tug.org/texlive/upgrade.html}{tug.org}
上面提供的说法是:
Windows 没有类似 Unix 的升级程序,
需要进行新安装%
\footnote{原文是: There is no comparable upgrade procedure for Windows.
Doing a new installation is necessary.}.

\section{升级宏包}\label{sec:windows:update}

安装完成后, 用户可以升级宏包以获得更好的使用体验. 
下面将介绍使用命令行升级宏包的方法. 
打开 \textsf{cmd} 窗口, 首先执行下面命令指定升级使用的镜像源. 
\texttt{ctan} 表示系统在升级时将自动寻求最近的源进行下载. 
\begin{lstlisting}[language=bash]
  tlmgr option repository ctan
\end{lstlisting}
用户同样可以指定其他的镜像源,
方法见附录~\ref{chp:appendix:mirror}.

接下来, 用户执行命令
\begin{lstlisting}[language=bash]
  tlmgr update --list
\end{lstlisting}
可查看目前源上可升级的宏包都有哪些. 
高级用户可以根据自己的需求选择升级特定宏包.
建议初级用户直接升级全部宏包. 
用户只需执行
\begin{lstlisting}[language=bash]
  tlmgr update --self --all
\end{lstlisting}
同时升级 \texttt{tlmgr} 本身和全部宏包. 

\subsection{\texttt{tlmgr} 本身无法成功升级}

遇到这种情况时, 用户需自行下载
\href{https://mirrors.ctan.org/systems/texlive/tlnet/update-tlmgr-latest.exe}{update-tlmgr-latest.exe},
然后再执行升级命令即可%
\footnote{这里依旧让系统自动选择镜像源,
用户可以自行选择国内镜像以加快下载速度}. 

\subsection{升级到一半停止}

这种情况下, 用户需要执行另外一个命令继续升级
\begin{lstlisting}
  tlmgr update --reinstall-forcibly-removed --all
\end{lstlisting}

\subsection{升级宏包的必要性}

自本手册撰写以来,
每年镜像文件打包时都可能存在一些问题,
具体讨论见%
\href{https://github.com/CTeX-org/ctex-kit/issues/569}{这里}.
用户安装完毕后,
请按照本节提供的方法升级.

\section{安装宏包}

在默认状态下, 用户将完整安装 \TeX{} Live, 因此用户极少碰到需要手动安装宏包的情形. 
同时, 在
\href{http://mirrors.ctan.org/info/lshort/chinese/lshort-zh-cn.pdf}{lshort-zh-cn}
中也明确提到, \textbf{如非万不得已, 尽量不要手动安装宏包}. 
因此在这里我只介绍从源处安装宏包的命令. 
假设用户想安装 \texttt{mcmthesis} 宏包, 只需在 \textsf{cmd} 执行
\begin{lstlisting}[language=bash]
  tlmgr install mcmthesis
\end{lstlisting}
需要注意的是, 用户一定要清楚所要安装的宏包名称, 并且在安装宏包前先确保镜像源设置正确.

\section{调出宏包手册}

当正确安装后,
用户可以调出宏包手册以查阅相应内容.
例如在 \textsf{cmd} 执行
\begin{lstlisting}[language=bash]
  texdoc texlive-zh-cn
\end{lstlisting}
或
\begin{lstlisting}[language=bash]
  texdoc lshort-zh-cn
\end{lstlisting}
就可分别调出 \texttt{texlive-zh-cn.pdf} 和 \texttt{lshort-zh-cn.pdf},
这两本手册中都有讲解安装的相关内容,
建议用户阅读.

本手册业已被 \TeX{} Live 收录,
调出本手册的命令为
\begin{lstlisting}[language=bash]
  texdoc install-latex-guide-zh-cn
\end{lstlisting}

\section{编译文档}\label{sec:windows:compile}

升级宏包完成后, 用户可以编译文档. 
这里依旧使用命令行来完成这一过程. 

首先, 用户需要在指定位置\footnote{以下称工作路径}建立一个 \texttt{tex} 文件:
\begin{lstlisting}[language = bash]
  mkdir D:\work-latex
  cd /d D:\work-latex
  notepad main.tex
\end{lstlisting}
第1行表示创建一个工作路径 \texttt{D:\textbackslash work-latex},
第2行表示进入工作路径, 第3行表示用记事本打开 \texttt{main.tex} 文件,
若文件不存在, 系统将询问用户是否创建该文件%
\footnote{Windows 11 对中文名变得友好一些了,
但这里不建议用户使用中文命名工作路径和文件名}.
在打开的记事本中编写一个最小示例
\begin{lstlisting}[language={[LaTeX]TeX}]
  \documentclass{article}
  \begin{document}
    Hello \LaTeX{} World!
  \end{document}
\end{lstlisting}
保存并退出. 
接下来执行
\begin{lstlisting}[language=bash]
  pdflatex main
\end{lstlisting}
等待系统完成编译过程. 
待编译完成后, 我们即可看到在 \texttt{D:\textbackslash work-latex}
中出现了 \texttt{main.pdf} 文件和其他同名的辅助文件
\texttt{main.aux} 与 \texttt{main.log}. 

对于中文文档, 可以在记事本中编写以下最小示例%
\footnote{自 Windows 10 1903 版本开始, 记事本默认使用 UTF-8 编码;
Windows 11 继承了这个特性.
若使用其他编辑器,
请注意规定文档编码为 UTF-8}%
\begin{lstlisting}[language={[LaTeX]TeX}]
  \documentclass[UTF8,fontset=windows]{ctexart}
  \begin{document}
    你好 \LaTeX{} 世界!
  \end{document}
\end{lstlisting}
保存并退出.
接下来执行
\begin{lstlisting}[language=bash]
  xelatex main
\end{lstlisting}
等待系统完成编译过程.
\texttt{xelatex} 编译命令配合 UTF-8 编码和 \texttt{ctex}
宏集已经是\textbf{目前主流的中文处理手段}.
\texttt{xelatex} 可以使用系统内安装的字体,
用户在安装字体时需右键字体文件的快捷菜单 \menu{显示更多选项 > 为所有用户安装}.

总结一下, 在使用 \LaTeX{} 时, 用户首先要在工作路径%
\footnote{建议不同的工程放入不同的工作路径}%
中建立一个完整的 \texttt{tex} 文件%
\footnote{该文件必须包含 \texttt{\textbackslash documentclass},
\texttt{\textbackslash begin\{document\}} 和
\texttt{\textbackslash end\{document\}},
这里不讨论包含子文件的情况,
文件名需为不含空格的英文}.
当 \texttt{tex} 文件内容确定后, 再保存文件并使用编译命令%
\footnote{目前常用的编译命令为 \texttt{pdflatex} 和 \texttt{xelatex}},
将 \texttt{tex} 文件编译成 \texttt{pdf} 文件. 

编译命令有很多可选参数, 如用户能够从 \texttt{pdf} 文件跳回 \texttt{tex} 文件,
便是因为执行编译时添加了参数
\begin{lstlisting}[language=bash]
  pdflatex -synctex=1 main
\end{lstlisting}
有些编译还需要额外调用系统命令,
例如使用 \textsf{minted} 包就有这类要求,
因此可以在执行编译时添加参数
\begin{lstlisting}[language=bash]
  pdflatex -shell-escape main
\end{lstlisting}
此外还有很多其他参数, 用户若感兴趣, 可自行阅读手册. 

在很多时候, 我个人更倾向于使用 \texttt{latexmk} 来编译 \texttt{tex} 文档,
如执行
\begin{lstlisting}[language=bash]
  latexmk -pdf -synctex=1 -interaction=nonstopmode main
\end{lstlisting}
意味着使用 
\begin{lstlisting}[language=bash]
  pdflatex -synctex=1 -interaction=nonstopmode main
\end{lstlisting}
来编译文档 \texttt{main.tex}, 并在有需要时完成其他步骤%
\footnote{如使用 \texttt{bibtex} 或 \texttt{biblatex} 处理参考文献时需要多次编译,
详情见相关文档}. 

\subsection{\LaTeXe 版本不匹配导致 \texttt{xelatex} 失败}

使用 \texttt{xelatex} 编译文档,
系统有时会提示目前 \LaTeXe 的版本低于某宏包的要求版本.
这种情况是由于 \textsf{fmt} 文件未能更新所导致.
此时可考虑在命令行运行
\begin{lstlisting}[language=bash]
  fmtutil-user --byfmt xelatex
\end{lstlisting}
待结束后再用 \texttt{xelatex} 编译文档.
更多内容参考
\href{https://github.com/CTeX-org/forum/issues/70}{github 上的讨论}.

另外也可考虑重装 \texttt{xetex}.
在命令行执行
\begin{lstlisting}[language=bash]
  tlmgr install --reinstall xetex
\end{lstlisting}

\subsection{编译出现 \texttt{l3backend-*} 问题}

这是 \texttt{fmt} 文件未能更新所导致的问题.
可以用
\begin{lstlisting}[language=bash]
  fmtutil-sys --all
\end{lstlisting}
刷新 \texttt{fmt} 文件后再编译 \texttt{tex} 文件.
% !TeX root = ../main.tex

\chapter{Ubuntu 18.04 系统}

\section{安装 \TeX{} Live}

这里只阐述如何用镜像安装.
为使用户顺利使用 \TeX{} Live 2019, 建议用户首先卸载从源内安装的 \TeX{} Live 的相关包.

下载 \href{http://mirrors.ctan.org/systems/texlive/Images/texlive2019.iso}{iso 镜像文件} 的方法如前所述, 可选择国内源以加快下载速度.
下载完毕后, 打开 \textsf{Terminal} 窗口, 执行以下命令
\begin{lstlisting}[language = bash]
  cd ~/Downloads
  md5sum texlive2019.iso
\end{lstlisting}
若显示
\begin{lstlisting}
  f13ffe81840bb37de855bf7445e1d29a  texlive2019.iso
\end{lstlisting}
则镜像文件下载正确.

接下来, 使用如下代码加载光盘镜像至 \texttt{\~{}/Downloads/texlive} 文件夹
\begin{lstlisting}[language = bash]
  mkdir ~/Downloads/texlive
  sudo mount ./texlive2019.iso ~/Downloads/texlive
\end{lstlisting}
接下来执行
\begin{lstlisting}[language = bash]
  sudo ~/Downloads/texlive/install-tl
\end{lstlisting}
进行安装.
在屏幕上应该能见到以下内容
\begin{lstlisting}
  ======================> TeX Live installation procedure <=====================
  
  ======>   Letters/digits in <angle brackets> indicate   <=======
  ======>   menu items for actions or customizations      <=======
  
   Detected platform: GNU/Linux on x86_64
   
   <B> set binary platforms: 1 out of 5
  
   <S> set installation scheme: scheme-full
  
   <C> set installation collections:
       40 collections out of 41, disk space required: 5845 MB
  
   <D> set directories:
     TEXDIR (the main TeX directory):
       /usr/local/texlive/2019
     TEXMFLOCAL (directory for site-wide local files):
       /usr/local/texlive/texmf-local
     TEXMFSYSVAR (directory for variable and automatically generated data):
       /usr/local/texlive/2019/texmf-var
     TEXMFSYSCONFIG (directory for local config):
       /usr/local/texlive/2019/texmf-config
     TEXMFVAR (personal directory for variable and automatically generated data):
       ~/.texlive2019/texmf-var
     TEXMFCONFIG (personal directory for local config):
       ~/.texlive2019/texmf-config
     TEXMFHOME (directory for user-specific files):
       ~/texmf
  
   <O> options:
     [ ] use letter size instead of A4 by default
     [X] allow execution of restricted list of programs via \write18
     [X] create all format files
     [X] install macro/font doc tree
     [X] install macro/font source tree
     [ ] create symlinks to standard directories
     [X] after install, set CTAN as source for package updates
  
   <V> set up for portable installation
  
  Actions:
   <I> start installation to hard disk
   <P> save installation profile to 'texlive.profile' and exit
   <H> help
   <Q> quit
  
  Enter command: 
\end{lstlisting}
这里强烈建议用户直接点击 \keys{I} 使用默认配置安装.
如果用户对于 Ubuntu 系统比较了解, 可以根据提示, 更改安装设置.
安装完毕后, 将加载的光盘镜像弹出
\begin{lstlisting}[language = bash]
  sudo umount ~/Downloads/texlive
\end{lstlisting}

默认安装完成后, 用户需要设置环境变量.
继续在 \textsf{Terminal} 中执行
\begin{lstlisting}[language = bash]
  sudo gedit ~/.bashrc
\end{lstlisting}
在打开的文件末尾添加
\begin{lstlisting}
  # Add Tex Live to the PATH, MANPATH, INFOPATH
  export PATH=/usr/local/texlive/2019/bin/x86_64-linux:$PATH
  export MANPATH=/usr/local/texlive/2019/texmf-dist/doc/man:$MANPATH
  export INFOPATH=/usr/local/texlive/2019/texmf-dist/doc/info:$INFOPATH
\end{lstlisting}
并保存退出.
这时再打开 \textsf{Terminal} 执行
\begin{lstlisting}[language=bash]
  tex -v
\end{lstlisting}
将显示
\begin{lstlisting}
  TeX 3.14159265 (TeX Live 2019)
  kpathsea version 6.3.1
  Copyright 2019 D.E. Knuth.
  There is NO warranty.  Redistribution of this software is
  covered by the terms of both the TeX copyright and
  the Lesser GNU General Public License.
  For more information about these matters, see the file
  named COPYING and the TeX source.
  Primary author of TeX: D.E. Knuth.
\end{lstlisting}
即为安装成功.

接下来处理字体.
首先将配置文件复制到系统,
在 \textsf{Terminal} 执行
\begin{lstlisting}[language=bash]
  sudo cp /usr/local/texlive/2019/texmf-var/fonts/conf/texlive-fontconfig.conf /etc/fonts/conf.d/09-texlive.conf
\end{lstlisting}
接下来刷新字体缓存,
继续在 \textsf{Terminal} 执行
\begin{lstlisting}[language=bash]
  sudo fc-cache -fsv
\end{lstlisting}
这样一来, \TeX{} Live 中的字体才能够被正确调用.

\section{卸载 \TeX{} Live}

如果要卸载从源内安装的 \TeX{} Live, 个人比较推荐使用 synaptic package manager.
\textsf{terminal} 中执行
\begin{lstlisting}[language = bash]
  sudo apt install synaptic
\end{lstlisting}
即可安装.
安装后打开, 搜索 \textsf{texlive} 即可看到与之相关的包, 右键标记以删除即可.

如果是从光盘镜像安装, 只需要直接删除文件夹即可.
可先在\textsf{terminal} 中执行
\begin{lstlisting}[language = bash]
  kpsewhich -var-value TEXMFROOT
\end{lstlisting}
来查询安装路径,
进而通过 \texttt{sudo rm -rf} 进行删除.
默认安装的用户直接运行
\begin{lstlisting}[language = bash]
  sudo rm -rf /usr/local/texlive/2019
  rm -rf ~/.texlive
\end{lstlisting}
当然删除后还要清理掉环境变量.

\section{跨版本升级 \TeX{} Live}
在 \href{https://www.tug.org/texlive/upgrade.html}{tug.org} 网站上提供了相应的方法.
但网站也声明:
默认情况下,
请通过执行新安装来获取新版本的 \TeX{} Live\footnote{原文是: By default, please get the new TL by doing a new installation instead of proceeding here.}.

\section{升级宏包}

首先在 \textsf{terminal} 中执行
\begin{lstlisting}[language = bash]
  sudo visudo
\end{lstlisting}
将
\begin{lstlisting}
  /usr/local/texlive/2019/bin/x86_64-linux:
\end{lstlisting}
添加在 \texttt{secure\_path} 中.
然后依次 \keys{\ctrl + X}, \keys{Y}, \keys{\enter} 保存退出.

接下来在 \textsf{terminal} 中执行
\begin{lstlisting}[language = bash]
  sudo tlmgr option repository ctan
\end{lstlisting}
更改源, \texttt{ctan} 也可以改成其他地址, 详情见 Windows 10 系统升级宏包部分.
接下来, 用户执行命令
\begin{lstlisting}[language = bash]
  sudo tlmgr update --list
\end{lstlisting}
可查看目前源上可升级的宏包都有哪些. 
高级用户可以根据自己的需求选择升级特定宏包, 而初级用户建议直接升级全部宏包. 
用户只需执行
\begin{lstlisting}[language = bash]
  sudo tlmgr update --self --all
\end{lstlisting}
同时升级 \texttt{tlmgr} 本身和全部宏包. 

\section{安装宏包}

Ubuntu 18.04 下安装宏包的要求与 Windows 10 下没有多少区别, 只需注意权限, 例如
\begin{lstlisting}[language = bash]
  sudo tlmgr install mcmthesis
\end{lstlisting}
即安装了 mcmthesis.

\section{编译文件}

首先, 用户需要在工作路径建立一个 \texttt{tex} 文件.
在 \textsf{Terminal} 中执行
\begin{lstlisting}[language = bash]
  mkdir ~/Documents/work_latex
  cd ~/Documents/work_latex/
  gedit main.tex
\end{lstlisting}
在打开的文件输入一个最小示例
\begin{lstlisting}[language = {[LaTeX]TeX}]
  \documentclass{article}
  \begin{document}
    Hello \LaTeX{} World!
  \end{document}
\end{lstlisting}
保存并退出. 
接下来执行
\begin{lstlisting}
  pdflatex main
\end{lstlisting}
等待系统完成编译过程. 
待编译完成后, 我们即可看到在 \texttt{\~{}/Documents/work\_latex} 中出现了 \texttt{main.pdf} 文件和其他同名的辅助文件 \texttt{main.aux} 与 \texttt{main.log}. 
执行
\begin{lstlisting}[language=bash]
  evince main.pdf
\end{lstlisting}
即可打开 \texttt{pdf} 文件.

编译命令可添加参数, 这里与 Windows 10 中的情形一致, 不再赘述.

\subsection{无法使用 \texttt{xelatex} 命令}

有些用户反映安装完毕后无法使用 \texttt{xelatex} 命令.
这里或许是因为 libfontconfig 缺失, 用户可在命令行中执行 
\begin{lstlisting}[language=bash]
  sudo apt-get install libfontconfig1
\end{lstlisting}
进行安装.

% !TeX root = ../main.tex

\chapter{Windows Subsystem for Linux}

目前微软推出了 Windows Subsystem for Linux (WSL) 供开发人员使用.
在这里简要介绍如何在 WSL 中安装 \TeX{} Live.
我选择的是微软商店中的 Ubuntu.
此篇安装教程仅供参考.
这里称 WSL 中使用的命令行为 \textsf{bash}.
\textbf{不建议}对操作系统了解不多的用户阅读本章内容.

\section{安装 \TeX{} Live}

在主系统%
\footnote{在 Windows 10 中直接进行的操作即为在主系统中的操作}%
中下载
\href{https://mirrors.ctan.org/systems/texlive/Images/texlive2020.iso}{iso 镜像文件},
可选择国内源以加快下载速度.
下载完毕后, 在 \textsf{cmd} 中验证文件是否正常.
换源下载及文件验证方法见 \ref{sec:windows:install}~节.

在正式安装前,
用户需要在 \textsf{bash} 中执行
\begin{lstlisting}[language=bash]
  sudo apt-get install libfontconfig1
  sudo apt-get install ttf-mscorefonts-installer
  sudo apt-get install fontconfig
\end{lstlisting}
这些命令是为了处理日后使用中可能出现的字体问题.
为避免安装速度过慢,
可仿照 \ref{subsec:ubuntu:xelatexfail}~节更改文件 \texttt{sources.list}.
注意 WSL 中无法启动 \textsf{gedit},
因此需要将其替换为 \textsf{vim},
即
\begin{lstlisting}[language=bash]
  sudo vim /etc/apt/sources.list
\end{lstlisting}
不熟悉 \textsf{vim} 的用户,
可参考 \ref{subsec:no-vim}~节.

接下来, 在主系统中将镜像挂载,
例如挂载到 \texttt{X:\textbackslash},
而后进入 \textsf{bash} 并执行如下命令
\begin{lstlisting}[language = bash]
  sudo mkdir /mnt/x
  sudo mount -t drvfs X: /mnt/x
\end{lstlisting}
如此便可在 \textsf{bash} 中找到挂载的光盘镜像.
之后执行
\begin{lstlisting}[language = bash]
  sudo /mnt/x/install-tl
\end{lstlisting}
进行安装.
在屏幕上应该能见到以下内容
\begin{lstlisting}
  ======================> TeX Live installation procedure <=====================

  ======>   Letters/digits in <angle brackets> indicate   <=======
  ======>   menu items for actions or customizations      <=======
  
  Detected platform: GNU/Linux on x86_64
  
  <B> set binary platforms: 1 out of 6
  
  <S> set installation scheme: scheme-full
  
  <C> set installation collections:
      40 collections out of 41, disk space required: 6516 MB
  
  <D> set directories:
    TEXDIR (the main TeX directory):
      /usr/local/texlive/2020
    TEXMFLOCAL (directory for site-wide local files):
      /usr/local/texlive/texmf-local
    TEXMFSYSVAR (directory for variable and automatically generated data):
      /usr/local/texlive/2020/texmf-var
    TEXMFSYSCONFIG (directory for local config):
      /usr/local/texlive/2020/texmf-config
    TEXMFVAR (personal directory for variable and automatically generated data):
      ~/.texlive2020/texmf-var
    TEXMFCONFIG (personal directory for local config):
      ~/.texlive2020/texmf-config
    TEXMFHOME (directory for user-specific files):
      ~/texmf

  <O> options:
    [ ] use letter size instead of A4 by default
    [X] allow execution of restricted list of programs via \write18
    [X] create all format files
    [X] install macro/font doc tree
    [X] install macro/font source tree
    [ ] create symlinks to standard directories
    [X] after install, set CTAN as source for package updates
  
  <V> set up for portable installation
  
  Actions:
  <I> start installation to hard disk
  <P> save installation profile to 'texlive.profile' and exit
  <H> help
  <Q> quit
  
  Enter command: 
\end{lstlisting}
用户直接点击 \keys{I} 使用默认配置安装.
如果用户对于 WSL 比较了解, 可以根据提示, 更改安装设置.
安装完毕后, 需继续在 \textsf{bash} 中执行
\begin{lstlisting}[language = bash]
  sudo umount /mnt/x
  sudo rmdir /mnt/x
\end{lstlisting}
弹出已加载的光盘镜像.

默认安装完成后, 用户需要设置环境变量.
继续在 \textsf{bash} 中执行
\begin{lstlisting}[language = bash]
  sudo vim ~/.bashrc
\end{lstlisting}
在打开的文件末尾添加
\begin{lstlisting}
  # Add TeX Live to the PATH, MANPATH, INFOPATH
  export PATH=/usr/local/texlive/2020/bin/x86_64-linux:$PATH
  export MANPATH=/usr/local/texlive/2020/texmf-dist/doc/man:$MANPATH
  export INFOPATH=/usr/local/texlive/2020/texmf-dist/doc/info:$INFOPATH
\end{lstlisting}
并保存退出.
同样, 不熟悉 \textsf{vim} 的用户可参考 \ref{subsec:no-vim}~节.
退出 WSL 再进入, 执行
\begin{lstlisting}[language=bash]
  tex -v
\end{lstlisting}
将显示
\begin{lstlisting}
  TeX 3.14159265 (TeX Live 2020)
  kpathsea version 6.3.2
  Copyright 2020 D.E. Knuth.
  There is NO warranty.  Redistribution of this software is
  covered by the terms of both the TeX copyright and
  the Lesser GNU General Public License.
  For more information about these matters, see the file
  named COPYING and the TeX source.
  Primary author of TeX: D.E. Knuth.
\end{lstlisting}
即为安装成功.

接下来仿照 Ubuntu 18.04 处理字体.
首先将配置文件复制到系统,
在 \textsf{Terminal} 执行
\begin{lstlisting}[language=bash]
  sudo cp /usr/local/texlive/2020/texmf-var/fonts/conf/texlive-fontconfig.conf /etc/fonts/conf.d/09-texlive.conf
\end{lstlisting}
接下来刷新字体缓存,
继续在 \textsf{Terminal} 执行
\begin{lstlisting}[language=bash]
  sudo fc-cache -fsv
\end{lstlisting}

\subsection{不懂 \textsf{vim} 的处理方法}\label{subsec:no-vim}

\textsf{vim} 的使用方法与目前很多流行的编辑器不同,
因此这里给出不用 \textsf{vim} 的处理方法.

首先更改 \texttt{sources.list}.
假设已经将 \texttt{sources.list} 备份,
而后在主系统中通过 \href{https://www.voidtools.com/zh-cn/}{everything}
搜索文件 \texttt{sources.list},
找到路径在 \menu{... > etc > apt} 的文件,
省略号表示 \menu{etc} 前那段很长的路径.
使用常用的编辑器打开 \texttt{sources.list} 并进行更改即可,
更改内容同 \ref{subsec:ubuntu:xelatexfail}~节.

接下来更改 \texttt{.bashrc}.
同样也是在主系统中通过
\href{https://www.voidtools.com/zh-cn/}{everything}
搜索文件 \texttt{.bashrc},
找到路径在 \menu{... > home > username} 的文件,
使用常用的编辑器打开并进行更改.

\section{卸载 \TeX{} Live}

直接删除文件夹即可.
先在 \textsf{bash} 中执行
\begin{lstlisting}[language = bash]
  kpsewhich -var-value TEXMFROOT
\end{lstlisting}
来查询安装路径,
进而通过 \texttt{sudo rm -rf} 进行删除.
默认安装的用户直接运行
\begin{lstlisting}[language = bash]
  sudo rm -rf /usr/local/texlive/2020
  rm -rf ~/.texlive2020
\end{lstlisting}
当然删除后还要清理掉环境变量.

\section{跨版本升级 \TeX{} Live}

WSL 中的方法同 \ref{sec:ubuntu:version}~节.

\section{升级宏包}

WSL 中的方法同 \ref{sec:ubuntu:update}~节.

\section{安装宏包}

WSL 中的方法同 \ref{sec:ubuntu:installpackage}~节.

\section{编译文件}

在这里, 我们假设已经有了一个最小示例 \texttt{main.tex}%
\footnote{可以在主系统中建立文件,
也可通过 \textsf{bash} 调用 \texttt{vim} 建立,
建立文件前需要确定工作路径},
内容为
\begin{lstlisting}[language = {[LaTeX]TeX}]
  \documentclass{article}
  \begin{document}
    Hello \LaTeX{} World!
  \end{document}
\end{lstlisting}
接下来在 \textsf{bash} 中进入工作路径,
执行
\begin{lstlisting}[language=bash]
  pdflatex main
\end{lstlisting}
等待系统完成编译过程. 
当然也可以在 \textsf{cmd} 中进入工作路径并执行
\begin{lstlisting}[language=bash]
  bash -i -c "pdflatex main"
\end{lstlisting}
实现编译.
待编译完成后, 可看到在工作路径中生成了 \texttt{main.pdf}
文件和其他同名的辅助文件 \texttt{main.aux} 与 \texttt{main.log}.
在主系统中可以打开 \texttt{main.pdf} 查看内容.

对于中文文档, 可以在主系统编写以下最小示例%
\footnote{注意使用 UTF-8 编码}%
\begin{lstlisting}[language={[LaTeX]TeX}]
  \documentclass[UTF8]{ctexart}
  \begin{document}
    你好 \LaTeX{} 世界!
  \end{document}
\end{lstlisting}
保存并退出.
接下来在 \textsf{bash} 中进入工作路径,
执行
\begin{lstlisting}[language=bash]
  xelatex main
\end{lstlisting}
等待系统完成编译过程.
同样可以在 \textsf{cmd} 中进入工作路径并执行
\begin{lstlisting}[language=bash]
  bash -i -c "xelatex main"
\end{lstlisting}
实现编译.
\texttt{xelatex} 同样可以使用 WSL 内安装的字体%
\footnote{注意主系统中的字体默认不能直接被调用,
如果打算使用主系统中的字体,
见
\href{https://github.com/OsbertWang/install-latex/issues/14}{github 上的讨论}},
在 WSL 中安装字体的方法见附录~\ref{chp:appendix:wsl}.

编译命令可添加参数, 这里与 \ref{sec:windows:compile}~节中的情形一致, 不再赘述.

\subsection{\LaTeXe 版本不匹配导致 \texttt{xelatex} 失败}

在 WSL 中将 \TeX{} Live 安装至默认路径时,
使用 \texttt{xelatex} 编译文档,
系统有时会提示目前 \LaTeXe 的版本低于某宏包的要求版本.
这种情况是由于 \textsf{fmt} 文件未能更新所导致.
此时可考虑在命令行运行
\begin{lstlisting}[language=bash]
  fmtutil-user --byfmt xelatex
\end{lstlisting}
待结束后再用 \texttt{xelatex} 编译文档.
更多内容参考
\href{https://github.com/CTeX-org/forum/issues/70}{github 上的讨论}.

另外也可考虑重装 \texttt{xetex}.
在命令行执行
\begin{lstlisting}[language=bash]
  sudo tlmgr install --reinstall xetex
\end{lstlisting}

\section{尚未圆满解决的问题}

将 \TeX{} Live 安装至 WSL 仍有悬而未决的问题.
由于 \textsf{bash} 默认是无窗口化的纯命令行,
因此用户无法直接通过命令 \texttt{texdoc} 打开相应的手册.
这里给出几种方法.

\subsubsection{explorer 浏览}

先找到手册所在路径,
例如寻找 \textsf{lshort-zh-cn}
\begin{lstlisting}[language=bash]
  texdoc -l lshort-zh-cn
\end{lstlisting}
系统会显示
\begin{lstlisting}
   1 /usr/local/texlive/2020/texmf-dist/doc/latex/lshort-chinese/lshort-zh-cn.pdf
     = [zh] The document itself
  Enter number of file to view, RET to view 1, anything else to skip:
\end{lstlisting}
这时按其他键 (如 \keys{x}) 退出.
然后分别执行
\begin{lstlisting}[language=bash]
  cd /usr/local/texlive/2020/texmf-dist/doc/latex/lshort-chinese/
  explorer.exe .
\end{lstlisting}
主系统的资源管理器便会打开.
这时用户便可以在主系统中使用 pdf 阅读器打开手册进行阅读.

\subsubsection{everything 搜索}

使用 \href{https://www.voidtools.com/zh-cn/}{everything}.
通过 \texttt{texdoc -l} 找到相应文件名,
而后直接搜索.
具体方法这里不再赘述.

\subsubsection{借助其他工具}

目前有很多开源的 WSL 辅助程序,
相关讨论见本手册的 \href{https://github.com/OsbertWang/install-latex/issues/13}{issue}.
% !TeX root = ../install-latex-guide-zh-cn.tex

\chapter{macOS}\label{chap:macOS}

\section{安装 Homebrew}

强烈建议用户使用 \href{https://brew.sh}{Homebrew}.
Homebrew 是一个包管理, 类似 Ubuntu 上面的 \texttt{apt-get}.
安装教程可以在其网站找到, 这里简单列出来:
\begin{lstlisting}[language=bash]
  /usr/bin/ruby -e "$(curl -fsSL https://raw.githubusercontent.com/Homebrew/install/master/install)"
\end{lstlisting}
将以上命令在\textsf{终端}\footnote{%
  打开方法为: \keys{\cmdmac + \SPACE}, 输入 \textsf{terminal} 并打开 \menu{终端} 应用}%
执行.
脚本会在执行前暂停, 并说明它将做什么. 依据屏幕指令执行即可.

中国大陆用户可以更改镜像以提高访问速度. 以中国科学技术大学镜像源为例:
\begin{lstlisting}[language=bash]
  cd "$(brew --repo)/Library/Taps/homebrew/homebrew-core"
  git remote set-url origin https://mirrors.ustc.edu.cn/homebrew-core.git
  cd "$(brew --repo)"/Library/Taps/homebrew/homebrew-cask
  git remote set-url origin https://mirrors.ustc.edu.cn/homebrew-cask.git
  echo 'export HOMEBREW_BOTTLE_DOMAIN=https://mirrors.ustc.edu.cn/homebrew-bottles' >> ~/.bash_profile
  source ~/.bash_profile
\end{lstlisting}
如果是 zsh 用户, 最后两行请替换为
\begin{lstlisting}[language=bash]
  echo 'export HOMEBREW_BOTTLE_DOMAIN=https://mirrors.ustc.edu.cn/homebrew-bottles' >> ~/.zshrc
  source ~/.zshrc
\end{lstlisting}

\subsection{Xcode}

根据 \href{https://docs.brew.sh/Xcode\#supported-xcode-versions}{Homebrew 网站}的提示\footnote{原文是 Homebrew supports and recommends the latest Xcode and/or Command Line Tools available for your platform},
推荐用户在安装 Homebrew 前先安装 Xcode.
在\textsf{终端} 执行以下命令即可:
\begin{lstlisting}[language=bash]
  xcode-select --install
\end{lstlisting}

\section{安装 Mac\TeX}

Mac\TeX{} 是 \TeX{} Live 在 macOS 下的再打包版本, 额外加入了一些辅助程序. 如果已经安装了 Homebrew,
只需在\textsf{终端}执行以下命令即可完成安装:
\begin{lstlisting}[language=bash]
  brew cask install mactex
\end{lstlisting}
如有输入密码等提示, 请根据屏幕指示操作.至于环境变量等繁琐细节, Homebrew 会自动进行处理,
无须用户干预.

完整的 Mac\TeX{} 会比较大. 如果磁盘空间实在紧张, 也可以考虑安装 Basic\TeX:
\begin{lstlisting}[language=bash]
  brew cask install basictex
\end{lstlisting}
安装完成后 Basic\TeX{} 依然会缺很多包, 手动安装会比较麻烦, 所以不推荐没有经验的用户尝试.

如果不通过 Homebrew 安装 Mac\TeX,
可以直接\href{https://mirrors.ctan.org/systems/mac/mactex/MacTeX.pkg}{下载}%
它的安装包而后手动安装.
同样可以考虑更换大陆镜像进行下载,
具体请参考附录~\ref{chp:appendix:mirror}.

\section{卸载 Mac\TeX}

如果用户借助 Homebrew cask 安装了 Mac\TeX,
那么卸载工作可能会稍显麻烦.
这里引用 \href{https://github.com/Homebrew/homebrew-cask/issues/32073}{Github} 上的讨论.
用户可以根据这里的内容卸载 Mac\TeX.

\section{跨版本升级 Mac\TeX}

跨版本升级 (Mac\TeX{} 的版本与 \TeX{} Live 保持一致), 可在\textsf{终端}借助 Homebrew 完成:
\begin{lstlisting}[language=bash]
  brew update
  brew cask upgrade mactex
\end{lstlisting}

\section{升级宏包}

升级宏包依旧可以使用 \texttt{tlmgr}.
使用方法与 \ref{sec:ubuntu:update}~节类似, 这里不再重复.
一般来说, 也需要使用 \texttt{sudo} 获取管理员权限后才能完成安装.

\section{安装宏包}

安装 CTAN 中的宏包方法与 \ref{sec:ubuntu:installpackage}~节一致.

\section{调出宏包手册}

调出宏包手册方法与 \ref{sec:ubuntu:texdoc}~节一致.

\section{编译文件}

假设已经用 TextEdit.app 或其他文本编辑器编写以下示例 \texttt{main.tex}%
\footnote{注意建立最小示例前先确定工作路径},
内容为
\begin{lstlisting}[language = {[LaTeX]TeX}]
  \documentclass{article}
  \begin{document}
    Hello \LaTeX{} World!
  \end{document}
\end{lstlisting}
接下来在\textsf{终端}中执行
\begin{lstlisting}[language=bash]
  pdflatex main
\end{lstlisting}
等待系统完成编译过程. 
待编译完成后, 可看到在工作路径中生成了 \texttt{main.pdf}
文件和其他同名的辅助文件 \texttt{main.aux} 与 \texttt{main.log}.
可以打开 \texttt{main.pdf} 查看内容.

对于中文文档, 可以编写以下最小示例%
\footnote{注意使用 UTF-8 编码}%
\begin{lstlisting}[language={[LaTeX]TeX}]
  \documentclass[UTF8]{ctexart}
  \begin{document}
    你好 \LaTeX{} 世界!
  \end{document}
\end{lstlisting}
保存并退出.
接下来在\textsf{终端}中进入工作路径,
执行
\begin{lstlisting}[language=bash]
  xelatex main
\end{lstlisting}
等待系统完成编译过程.
\texttt{xelatex} 可调用系统字体,
为系统安装字体的方法请参考
\href{https://support.apple.com/en-us/HT201749}{How to install and remove fonts on your Mac}.
安装完成后, 刷新字体缓存.
注意到此时 \texttt{xelatex} 只能通过文件名来调用发行版预装的字体,
解决此问题的方法请参考%
\href{https://zhuanlan.zhihu.com/p/59774395}{慕子的文章}.

编译命令的相关参数, 这里不再赘述.

% !TeX root = ../install-latex-guide-zh-cn.tex

\chapter{Overleaf}

在特定场合,
有些用户并不需要也没条件在本地安装发行版,
因此这里额外补充 \href{www.overleaf.com}{Overleaf} 的相关内容.

\section{注册 Overleaf}

Overleaf 是全球范围内首屈一指的在线 \LaTeX{} 编辑平台.
它为每位用户提供了 Ubuntu 系统下的 \TeX{} Live.
它优秀的协作功能、丰富的模板仓库已吸引全球科研工作者成为它的用户.
Overleaf 提供了包括%
\href{https://cn.overleaf.com}{中文}%
在内的多种语言供用户使用.
2020年10月6日,
\href{https://www.overleaf.com/blog/tex-live-2020-now-available}{它将后台的 \TeX{} Live 升级为 2020 版本},
同时,
\href{https://www.overleaf.com/blog/new-feature-select-your-tex-live-compiler-version}{Overleaf 还允许用户自主选择项目中的 \TeX{} Live 版本}.

目前 Overleaf 允许用户%
\href{https://www.overleaf.com/learn/latex/Chinese}{使用中文},
并为用户预先准备了一些%
\href{https://www.overleaf.com/learn/latex/Questions/What_OTF/TTF_fonts_are_supported_via_fontspec%3F#Fonts_for_CJK}{中文字体}.
这些中文字体可通过%
\href{https://www.overleaf.com/latex/templates/using-the-ctex-package-on-overleaf-zai-overleafping-tai-shang-shi-yong-ctex/gndvpvsmjcqx}{C\TeX{} 宏集}%
调用.
特别推荐用户使用%
\href{https://www.overleaf.com/latex/examples/demonstration-of-noto-serif-cjk-and-noto-sans-cjk-fonts/sgrwgcddtqsq}{思源宋体和思源黑体}%
这两种开源中文字体.


遗憾的是,
国内网络环境会对 Overleaf 所使用的 reCaptcha 造成影响.
这也使得很多用户在直接注册 Overleaf 时就遇到了问题.

目前,
比较好的替代方案是借助 \href{https://orcid.org}{ORCID} 来进行注册.
目前使用国内网络访问 ORCID 还比较流畅.
用户, 尤其是科研工作者, 可以先注册一个 ORCID 账号.
未来投稿时,
可将 ORCID 账号与自己的期刊网站账号进行绑定.
同时,
用户也可逐步将自己所发表论文列在 ORCID 网站以便管理.

\section{使用 Overleaf}

用户通过 ORCID 注册 Overleaf 后便可进入自己的项目列表页面进行使用.

新建项目是用户首先使用的功能.
Overleaf 提供了多种渠道为用户新建项目:
可以通过 Overleaf 中的模板,
也可以通过上传本地的 \textsf{zip} 压缩文件包.

新建项目后,
用户便可进入编辑界面.
在编辑界面用户需要先在左上角 \menu{Menu} 中选择合适的编译命令.
由于默认字体和文件编码等原因,
强烈建议用户在处理中文文档时使用 \texttt{XeLaTeX}.
编写文档后,
用户可通过鼠标点击按钮进行编译,
也可使用快捷键 \keys{ctrl + enter}.

用户编写的文件会保存在网站.
编写完成后,
用户只需点击 \menu{Menu} 旁的箭头回到项目列表.
这时可以看到新增项目右侧有四个图标,
它们分别是 \menu{Copy}、\menu{Download}、\menu{Archive} 和 \menu{Trash}.
用户可根据自己的需求点击合适的图标.

\section{选择不同发行版版本}

Overleaf 将后台 \TeX{} Live 升级后,
用户新建项目默认使用 \TeX{} Live 2020,
而老项目还是使用 \TeX{} Live 的老版本.
如果用户打算使用 \TeX{} Live 2020,
只需在 \menu{Menu > Settings > TeX Live version} 选择版本即可.

\section{学习与帮助}

Overleaf 的%
\href{https://www.overleaf.com/latex/templates}{模板}和%
\href{https://www.overleaf.com/learn}{文档}%
对全网公开,
用户可以自行学习.
另外 Overleaf 有着专业的技术支援团队,
用户可发送邮件至 \href{mailto:support@overleaf.com}%
{\ttfamily support@overleaf.com}
咨询使用过程中遇到的问题,
在邮件中请注意文明用语.
部分问题会超出免费服务的范畴,
用户需谨记这点.
\chapter{总结和展望}

本文是个人最近一段时间的使用总结, 其中难免有不甚合理或晦涩难懂的部分. 
若用户在阅读本文档的过程中有任何意见和建议, 请发邮件或在 GitHub 中提 issue.

有用户指出,
WSL 中安装字体比较麻烦.
这里引用 \href{https://www.jianshu.com/p/e7f12b8c8602}{Ubuntu系统字体命令和字体的安装} 一文,
希望能够提供一些帮助.

首先获取需要安装的字体文件,
假设文件保存在 \verb|~/fonts/|.
然后在 \texttt{/usr/share/fonts/} 文件夹中创建新的文件夹,
例如 \texttt{myfonts}
\begin{lstlisting}[language=bash]
  cd /usr/share/fonts/
  sudo mkdir myfonts
\end{lstlisting}
接下来将获取的字体文件复制到 \texttt{myfonts} 中
\begin{lstlisting}[language=bash]
  sudo cp ~/fonts/* /usr/share/fonts/myfonts/ 
\end{lstlisting}
然后修改字体文件的权限
\begin{lstlisting}[language=bash]
  sudo chmod -R 755 myfonts
\end{lstlisting}
最后建立字体缓存
\begin{lstlisting}[language=bash]
  mkfontscale
  mkfontdir
  fc-cache -fv
\end{lstlisting}

在目前的版本中, 本文使用 WSL 方法进行编译.
目前感受是该方法较之 Windows 10 系统中的编译更快.
其他方面并未进行比较.
希望阅读本文的用户能够尽快上手.


\end{document}
